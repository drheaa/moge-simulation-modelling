%=========================================================
% CHAPTER 4: METHODOLOGY — EM ALGORITHM OVERVIEW
%=========================================================


\chapter{Methodology: EM Algorithm Overview}
\label{chap:em_overview}


The Expectation--Maximization (EM) algorithm, introduced by
Dempster, Laird and Rubin (1977), is a widely used iterative method
for obtaining maximum likelihood estimates (MLEs) in the presence of
incomplete or latent data structures. Many statistical models, including
members of the Marshall--Olkin family, naturally involve mechanisms that
cannot be fully observed. In such cases, the observed-data likelihood
becomes analytically complex, whereas a corresponding ``complete-data''
formulation is significantly simpler.


This chapter provides a conceptual overview of the EM algorithm, the
conditions under which it is used, and why it is particularly suitable
for the Marshall--Olkin Generalized Exponential (MOGE) distribution.
No mathematical derivations appear here; these are presented in the
next chapter.


%---------------------------------------------------------
\section{Conceptual Overview of the EM Algorithm}
%---------------------------------------------------------


When maximum likelihood estimation is performed under missing or
unobserved information, the log-likelihood often becomes difficult to
maximize directly. The key idea of the EM algorithm is to treat the
data as if it consists of two parts: an observed component and an
unobserved (latent) component. If the full data were available,
maximization would typically be straightforward. The EM algorithm
leverages this by iteratively ``filling in'' the missing part through
conditional expectations.


Each iteration of EM has two steps:


\begin{itemize}
    \item \textbf{E-step (Expectation):}  
    Compute the expected value of the complete-data log-likelihood
    with respect to the conditional distribution of the latent variables,
    given the observed data and current parameter estimates.


    \item \textbf{M-step (Maximization):}  
    Maximize this expected log-likelihood with respect to the model
    parameters to obtain updated estimates.
\end{itemize}


This two-step procedure is repeated until convergence, meaning that
successive parameter estimates change negligibly.


%---------------------------------------------------------
\section{When EM Is Used}
%---------------------------------------------------------


The EM algorithm is appropriate in a wide range of settings, including:


\begin{itemize}
    \item \emph{Incomplete data scenarios}, where some components of the
    data-generating process are unobserved.
    \item \emph{Latent variable models}, such as mixture models, shock
    models, and failure-time models with unobserved causes.
    \item \emph{Censored or truncated data}, common in reliability and
    survival analysis.
    \item \emph{Likelihoods with no closed-form maximizers}, where solving
    the likelihood equations directly is either impossible or unstable.
\end{itemize}


In these cases, the observed-data likelihood may involve complicated
integrals or high-dimensional nonlinear systems that cannot be solved
analytically. EM simplifies the optimization by replacing the missing
components with their conditional expectations.


%---------------------------------------------------------
\section{Latent Variables in EM}
%---------------------------------------------------------


Latent variables represent unobserved structural features of the model.
For the MOGE distribution, the formulation introduced by Risti\'c and
Kundu (2015) shows that the model can be expressed using an unobserved
quantity (often denoted as $Z$) representing an underlying geometric or
shock-based mechanism. Incorporating $Z$ transforms the observed-data
log-likelihood into a much simpler complete-data log-likelihood.


Although $Z$ is not observable, its conditional expectation
$E(Z \mid X)$ can be computed explicitly. This makes the E-step
tractable and leads to separable maximization steps in the M-step.


%---------------------------------------------------------
\section{Steps of the EM Algorithm}
%---------------------------------------------------------


\subsection*{E-Step: Estimating the Missing Information}
Given parameter values
$(\alpha^{(k)}, \lambda^{(k)}, \theta^{(k)})$, the E-step computes:
\[
E\!\left[ Z \mid X ; \alpha^{(k)}, \lambda^{(k)}, \theta^{(k)} \right],
\]
and constructs the expected complete-data log-likelihood. This
``pseudo'' log-likelihood treats the missing structure as known but
replaces it with its conditional expectation.


\subsection*{M-Step: Updating the Parameters}
The M-step maximizes the pseudo log-likelihood with respect to
$(\alpha, \lambda, \theta)$.  
For MOGE, this step becomes significantly simpler than maximizing the
original likelihood, because the latent structure allows the log-likelihood
to decompose into parts that can be optimized separately.


\subsection*{Iteration}
The updated parameters are returned to the next E-step. Convergence is
typically assessed by checking whether
\[
\left| \ell^{(k+1)} - \ell^{(k)} \right|
\]
or the relative parameter changes fall below a chosen tolerance.


%---------------------------------------------------------
\section{Why EM Provides Stable Updates}
%---------------------------------------------------------


The EM algorithm is known for its computational stability:


\begin{itemize}
    \item Each iteration is guaranteed not to decrease the observed-data
    log-likelihood.
    \item Parameter updates are smooth and avoid the large, unstable jumps
    common in Newton--Raphson or quasi-Newton methods.
    \item EM does not require second derivatives, reducing numerical
    sensitivity.
\end{itemize}


These properties make EM especially attractive for models whose likelihood
surfaces are complicated or nearly flat in certain directions, which is
typical for Marshall--Olkin type models.


%---------------------------------------------------------
\section{Why EM Is Needed for the MOGE Model}
%---------------------------------------------------------


For the MOGE distribution, the likelihood equations for
$(\alpha, \lambda, \theta)$ do \emph{not} admit closed-form solutions.
Risti\'c and Kundu (2015) showed that:


\begin{itemize}
    \item The observed-data log-likelihood involves nonlinear expressions
    such as
    \[
        \left( \theta + (1-\theta)(1-e^{-\lambda x})^{\alpha} \right)^{-2},
    \]
    which make the score equations analytically intractable.


    \item Direct numerical optimization requires solving a
    three-dimensional nonlinear system, which is computationally unstable
    and highly sensitive to starting values.


    \item Introducing the latent variable $Z$ leads to a complete-data
    log-likelihood that is far easier to optimize, allowing the estimation
    problem to be separated into a sequence of one-dimensional tasks.
\end{itemize}


Thus, the EM framework is not merely convenient but essential: it provides
a practical and stable method for computing the MLEs of the MOGE parameters.


%---------------------------------------------------------
\section{Summary}
%---------------------------------------------------------


This chapter presented a conceptual overview of the EM algorithm and
explained why it is the appropriate estimation method for the
Marshall--Olkin Generalized Exponential distribution. EM allows the
complex observed-data likelihood to be replaced with a tractable
complete-data formulation, enabling stable and efficient parameter
estimation. The next chapter develops the full EM derivation for the
MOGE model, including the complete-data structure, the conditional
expectations in the E-step, and the explicit update equations used in  
the M-step.
