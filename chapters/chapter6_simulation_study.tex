\chapter{Data Analysis}

\section{Introduction}

In this chapter, we analyze a real tensile-strength dataset consisting of 56 single carbon fiber measurements tested under tension at a gauge length of 1 mm, recorded in GPa. The dataset, originally provided by Prof. R.G. Surles, represents the tensile strength of individual fibers extracted from a 1000-filament tow. Since tensile strength is strictly positive and continuous, it is well-suited for lifetime distribution modeling often used in reliability and failure-time analysis.

The objective of this analysis is to evaluate how well the Marshall--Olkin Generalized Exponential (MOGE) distribution models the strength characteristics relative to traditional reliability distributions, namely Weibull and Gamma. Due to the three-parameter flexibility of MOGE $(\alpha, \lambda, \theta)$, particularly in handling increasing hazard structures and heavy-tailed behavior, parameter estimation is carried out using the EM algorithm. In this chapter, exploratory visualization, model fitting, goodness-of-fit testing, and comparative inference are presented. The results here provide the real-data foundation leading to simulation studies discussed later in Chapter 7.

% ----------------------------------------------------------
\section{Exploratory Data Analysis}
% ----------------------------------------------------------

\subsection*{Histogram with Fitted MOGE PDF}

\begin{figure}[H]
    \centering
    \includegraphics[width=0.75\textwidth]{figures/hist_moge.png}
    \caption{Histogram of Carbon Fiber Strength with fitted MOGE PDF}
\end{figure}

The histogram shows a unimodal distribution centred near 3.8--4.5 GPa with slight right-skewness, indicating most fibers fail in a moderate strength region while a smaller fraction withstand higher loads. The fitted MOGE PDF aligns well with the observed frequencies, capturing both the peak density and gradual tail descent. No multimodality or extreme outliers are observed, confirming suitability for parametric modelling through reliability distributions. The presence of a right tail suggests increasing hazard behaviour, motivating the use of models capable of shape control such as MOGE, Weibull, and Gamma.

% ----------------------------------------------------------
\subsection*{Empirical CDF vs MOGE CDF}

\begin{figure}[H]
    \centering
    \includegraphics[width=0.75\textwidth]{figures/cdf_moge.png}
    \caption{Empirical CDF vs Fitted MOGE CDF}
\end{figure}

Figure 6.2 compares the empirical cumulative distribution with the theoretical MOGE CDF. The two curves follow each other closely with minimal deviation across the support range. A low Kolmogorov--Smirnov statistic of \textbf{0.0474} confirms strong agreement. Minor divergence appears only in the extreme upper tail due to data sparsity, while overall tracking remains smooth and consistent. This alignment validates the capability of MOGE in capturing cumulative failure behavior relevant for reliability thresholds.

% ----------------------------------------------------------
\subsection*{TTT Plot (Total Time on Test)}

\begin{figure}[H]
    \centering
    \includegraphics[width=0.7\textwidth]{figures/ttt_plot.png}
    \caption{TTT plot of Carbon Fiber Strength Data}
\end{figure}

The empirical TTT curve lies below the 45° reference line, indicating an \textbf{Increasing Failure Rate (IFR)}. This behaviour is typical of brittle materials where probability of breakage rises as stress increases. The IFR pattern further justifies using flexible lifetime models beyond exponential, aligning with properties of MOGE. Mathematically,

\[
\text{TTT curve below diagonal} \Rightarrow \text{Increasing Failure Rate (IFR)}
\]

% ----------------------------------------------------------
\section{Model Fitting and Parameter Estimation}

\begin{figure}[H]
    \centering
    \includegraphics[width=0.8\textwidth]{figures/pdf_comparison.png}
    \caption{PDF comparison: MOGE vs Weibull vs Gamma}
\end{figure}

Three lifetime distributions (MOGE, Weibull, and Gamma) were fitted to the dataset. Parameter estimation for Weibull and Gamma used MLE, while MOGE parameters $(\alpha, \lambda, \theta)$ were estimated using EM due to latent-variable structure. The PDFs show that MOGE closely follows the empirical shape, capturing central mass and tail behaviour more accurately. Weibull decays faster in the right tail, while Gamma shows mid-range underfitting. This visual evidence points toward superior flexibility of MOGE.

% ----------------------------------------------------------
\section{Goodness-of-fit Results}

Goodness-of-fit was assessed through log-likelihood and KS metrics. The fitted results obtained are:

\[
\text{MOGE: } \alpha=201.346,\; \lambda=2.0817,\; \theta=33.529,\; LL=-68.00,\; KS=0.0474
\]
\[
\text{Weibull: shape}=5.706,\; scale=4.596,\; LL=-68.93,\; KS=0.0902
\]
\[
\text{Gamma: shape}=26.284,\; scale=0.162,\; LL=-68.38,\; KS=0.0537
\]

MOGE achieves the highest log-likelihood and lowest KS statistic, confirming it best describes the data. Weibull performs moderately with tail deviation, while Gamma mismatches the centre region slightly. Overall, statistical metrics and curve-fitting both verify MOGE as the most appropriate model.

% ----------------------------------------------------------
\section{Discussion}

The results clearly show the capability of MOGE to model carbon fiber strength more flexibly than Weibull and Gamma. The superior fit arises from its third parameter $\theta$, which governs tail behaviour and hazard adaptability. The IFR pattern observed in the TTT plot supports suitability of MOGE, reflecting realistic brittle failure dynamics where breakage probability rises near threshold load. PDF/CDF visual overlays further strengthen this conclusion, showing strong alignment across quantiles.

% ----------------------------------------------------------
\section{Conclusion}

This chapter demonstrated successful application of lifetime modelling to carbon fiber strength. Among the models tested, MOGE provided the best representation, as shown by its lowest KS statistic, highest likelihood, and strongest visual fit. Weibull and Gamma offered reasonable alternatives but lacked adaptability in tail and mid-range fitting. MOGE enables more reliable inference in materials engineering settings where extreme-strength behavior matters. These findings motivate further simulation and robustness evaluation in Chapter 7.

