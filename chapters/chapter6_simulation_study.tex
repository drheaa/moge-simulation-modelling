\chapter{Data Analysis}

\section{Introduction}

The real dataset used in this study consists of first-goal times recorded in FIFA Club World Cup matches (2025 season). The data was sourced from FootyStats, a publicly available football statistics platform. Each observation represents the minute at which the first goal was scored by the home or away team. Matches in which no goals were scored were excluded, as the model assumes strictly positive event times.

After the data was preprocessed and deduplicated, 29 observations each were obtained for home and away goal timings. Since these values are strictly positive, continuous, and time-to-event in nature, they are suitable for modelling using survival/lifetime distributions such as Weibull, Gamma, and the Marshall–Olkin Generalized Exponential (MOGE).

The purpose of this chapter is to examine the behaviour of real scoring data, identify which probability distribution best describes it, and assess whether MOGE is a suitable modelling option for football scoring. Insights derived here form the basis for the simulation work in Chapter 7.

% ---------------------------------------------------------
\section{Exploratory Data Analysis}

\subsection{Histogram of First Goal Times}

\begin{figure}[h!]
\centering
\includegraphics[width=0.8\linewidth]{fig6_1_histogram.png}
\caption{Histogram of First Goal Times in FIFA Club World Cup Matches}
\end{figure}

The histogram of first-goal timings reveals that most goals occur within the first 30–40 minutes of the match. A long right-tail extends beyond this range, signalling the possibility of late breakthroughs, though occurring less frequently. The distribution is strongly right-skewed and lacks multimodal structure, suggesting that scoring does not cluster but instead follows a continuous waiting process. This supports the use of survival-based models and suggests that the hazard rate is unlikely to remain constant over time.

\subsection{Empirical CDF – Home vs Away First Goals}

\begin{figure}[h!]
\centering
\includegraphics[width=0.8\linewidth]{fig6_2_ecdf.png}
\caption{Empirical CDF for Home and Away First-Goal Times}
\end{figure}

The empirical cumulative distribution functions (ECDFs) show that both home and away teams reach around 80–90\% probability of scoring by the 40th minute. The curves rise sharply during the initial phase and flatten thereafter, indicating that early scoring is common while late scoring is sporadic. Away teams appear to score slightly earlier on average, though the curves converge later in the match.

The strong concave rise followed by flattening contradicts the constant hazard assumption of an exponential model and supports more flexible distributions such as Weibull, Gamma and MOGE. The weak negative correlation between $X_1$ (home) and $X_2$ (away) further suggests that scoring processes behave independently, justifying separate modelling.

% ---------------------------------------------------------
\subsection{TTT Plot (Total Time on Test)}

\begin{figure}[h!]
\centering
\includegraphics[width=0.8\linewidth]{fig6_3_ttt_home.png}
\includegraphics[width=0.8\linewidth]{fig6_4_ttt_away.png}
\caption{TTT plots for Home and Away First-Goal Times}
\end{figure}

The Total-Time-on-Test plots lie below the diagonal reference line, indicating an Increasing Failure Rate (IFR). This means the likelihood of scoring increases as the match progresses. Home teams show slightly stronger curvature than away teams, suggesting greater attacking momentum as the match unfolds. Away scoring grows more gradually, possibly due to tactical conservatism or defensive setups early in matches.

\[
\text{TTT Curve Below Diagonal} \Rightarrow \text{Increasing Hazard Rate (IFR)}
\]

The IFR trend reinforces the suitability of non-memoryless distributions such as Weibull and Gamma, and theoretically also MOGE, for modelling first-goal timings.

% ---------------------------------------------------------
\section{Model Fitting and Parameter Estimation}

Three lifetime distributions – Weibull, Gamma and MOGE – were fitted to both $X_1$ and $X_2$ datasets. Weibull and Gamma parameters were estimated using Maximum Likelihood Estimation, whereas MOGE required the Expectation-Maximization (EM) algorithm due to its latent variable structure and three-parameter formulation $(\alpha,\lambda,\theta)$.

Model suitability was evaluated using log-likelihood, Kolmogorov–Smirnov (KS) goodness-of-fit statistics and Akaike Information Criterion (AIC), allowing comparison based on distributional fit while accounting for model complexity.

% ---------------------------------------------------------
\section{Goodness-of-Fit Results}

Goodness-of-fit results revealed that MOGE performed poorly for real data, yielding KS statistics close to 1 with extremely low p-values, leading to strong rejection for both home and away timings. The EM estimator also showed instability, likely due to the small sample size and irregular scoring patterns.

In contrast, the Weibull distribution achieved the closest fit to empirical scoring behaviour. Gamma also performed well by capturing skewness and tail behaviour, while the Exponential model performed poorly due to its constant hazard assumption.

These results indicate that Weibull and Gamma are better suited for real scoring data than MOGE. However, the poor performance of MOGE appears to stem from data limitations rather than theoretical inadequacy. This motivates simulation-based evaluation of MOGE under ideal conditions.

% ---------------------------------------------------------
\section{Discussion}

The real-data analysis confirms that first-goal arrivals follow a time-dependent scoring process with increasing hazard. The right-skewed distribution, early scoring dominance and IFR behaviour observed through the ECDF and TTT plots align with survival-based scoring interpretations. Weibull and Gamma models performed best empirically, while MOGE struggled due to instability under limited data.

Although MOGE is theoretically flexible and capable of representing complex hazard shapes, the small dataset restricts its estimation stability. With larger or multi-season datasets, MOGE performance may improve substantially. To validate this hypothesis, a simulation study was designed to evaluate estimator behaviour, bias, mean squared error and convergence properties using synthetic data (Chapter 7).

% ---------------------------------------------------------
\section{Conclusion of Real Data Analysis}

Real first-goal times show heavy concentration in the early match period followed by a long right tail. Hazard rate increases over time, confirming non-memoryless behaviour. ECDF analysis highlights early scoring likelihood, while TTT plots verify IFR behaviour for both home and away teams. Weak correlation between scoring processes supports independent modelling.

Model fitting results show Weibull as the best performing model, followed by Gamma, whereas MOGE was rejected statistically. This does not dismiss the potential of MOGE; instead, it reflects sample limitations. These findings motivated the simulation work in Chapter 7, where synthetic data is used to evaluate MOGE under controlled settings and examine its theoretical strengths.

