\chapter{Simulation Study}

\section{Introduction}

In this chapter, we conduct a simulation study to assess the performance of the EM algorithm for the MOGE distribution under controlled conditions. Unlike real data analysis, simulation allows us to define true parameters and generate synthetic observations directly from the MOGE distribution. This removes external variability and makes it possible to evaluate how well the EM algorithm performs across varying sample sizes.

% ---------------------------------------------------------
\section{Simulation Setup and Methodology}

Synthetic i.i.d.\ samples were generated from the MOGE distribution with true parameters:
\[
(\alpha_0 = 1.5,\; \lambda_0 = 0.8,\; \theta_0 = 1.2),
\]
chosen to reflect moderately increasing hazard rates.

Samples were generated for increasing sample sizes:
\[
n \in \{30,\, 50,\, 100,\, 200\}.
\]

For each sample size, multiple Monte Carlo replications were performed. In each run, synthetic times were drawn from the MOGE model, the EM algorithm was applied to re-estimate parameters, and resulting estimates were recorded.

% ---------------------------------------------------------
\section{Estimator Performance}

The distribution of parameter estimates for each sample size was visualized. When $n=30$, estimates were widely spread, indicating high variance and uncertainty. As sample size increased to $n=50$ and $n=100$, the dispersion narrowed considerably. By $n=200$, the estimator became tightly concentrated.

The mean squared error and bias were calculated for each estimated parameter. Both metrics displayed a clear decreasing trend across increasing sample sizes, confirming that the EM procedure for MOGE is asymptotically consistent.

% ---------------------------------------------------------
\section{Convergence Behaviour}

The log-likelihood values were recorded across EM iterations. As expected from the EM algorithm theory, the likelihood improved after each update step, reflecting the non-decreasing likelihood guarantee.

In full multi-iteration runs, the likelihood continued to increase until reaching stability, confirming convergence occurred smoothly without oscillation or divergence.

% ---------------------------------------------------------
\section{Summary}

The simulation study demonstrates that the MOGE EM estimator is theoretically sound and performs well under controlled conditions. The model is capable of recovering true parameters with increasing accuracy as sample size grows.
