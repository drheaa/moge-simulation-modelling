\chapter{Introduction}

Lifetime and reliability data appear frequently in engineering, medicine, survival
studies, and industrial applications. Classical models such as the Exponential,
Weibull, or Gamma distributions are commonly used due to their mathematical
tractability and interpretability. However, in practice, these classical models fail
to capture many important hazard-rate shapes, especially non-monotonic failure
patterns. Many real-world systems exhibit bathtub-shaped or inverted bathtub
hazard functions—patterns that traditional exponential-type models cannot
adequately represent.

To overcome these limitations, Marshall and Olkin (1997) introduced a general
method for adding an extra parameter to an existing distribution family. Their
construction allows greater flexibility while retaining mathematical tractability.
Building on this idea, Ristić and Kundu (2015) proposed the
\textit{Marshall–Olkin Generalized Exponential (MOGE)} distribution. This model
extends the two-parameter Generalized Exponential (GE) distribution by
introducing an additional parameter $\theta$, resulting in a more flexible
three-parameter family.

The MOGE distribution is capable of generating a wide range of density shapes
and supports all four primary hazard-rate behaviours:

\begin{itemize}
    \item increasing,
    \item decreasing,
    \item bathtub-shaped,
    \item upside-down bathtub shaped.
\end{itemize}

Because of this versatility, the MOGE model is a valuable tool for analyzing
complex lifetime data where simpler models fail. Despite its flexibility, the
distribution maintains a compact and tractable analytical form, making it suitable
for parameter estimation and for modeling censored lifetime data.

This study revisits and summarizes the theoretical structure of the MOGE
distribution, focusing particularly on parameter estimation using the
Expectation–Maximization (EM) algorithm. In later chapters, we present a
simulation study to evaluate the performance of the EM algorithm for estimating
the model parameters under various parameter settings and sample sizes.
