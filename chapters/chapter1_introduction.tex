\chapter{Introduction}

Engineering, medical studies, and survival analysis all revolve around the modelling of
lifetime and reliability data. While classical models such as the Exponential, Weibull,
and Gamma distributions are mathematically convenient, they often fail to capture the
non-monotonic hazard trends observed in real systems \citep{chugani2025a,chugani2025b}.
Many practical scenarios exhibit early-failure phases, wear-out periods, or complex
bathtub-shaped hazard behaviour that require more flexible distributional forms
\citep{gfg2025}.

To address these limitations, Marshall and Olkin \citep{marshall1997} proposed a general
method for enriching an existing distribution by introducing an additional parameter.
Building on this framework, Risti\'c and Kundu \citep{ristic2015} developed the
Marshall--Olkin Generalized Exponential (MOGE) distribution, a three-parameter extension
of the Generalized Exponential model capable of representing all major hazard-rate
shapes. Its tractability and flexibility make the MOGE distribution a promising candidate
for modelling complex lifetime data.

Despite its advantages, estimating MOGE parameters is challenging because the likelihood
equations do not admit closed-form solutions. The Expectation--Maximization (EM)
algorithm provides a natural solution by employing a latent-variable representation that
simplifies the optimization into a manageable sequence of updates \citep{gfg2019}.
This study presents the full EM framework for the MOGE distribution, investigates its
statistical characteristics through simulation, and examines its practical behaviour
across varying sample sizes. Particular emphasis is placed on assessing the strengths and
limitations of the estimation procedure, especially in small-sample settings.
