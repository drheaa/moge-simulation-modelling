\chapter{Conclusion}

This study examined the Marshall–Olkin Generalized Exponential (MOGE)
distribution, a flexible extension of the classical Generalized Exponential model.
By introducing the additional shape parameter $\theta$ through the
Marshall–Olkin method, the MOGE distribution is capable of capturing a broad
range of lifetime behaviours, including increasing, decreasing, bathtub-shaped,
and inverted-bathtub hazard functions.

\section*{Major Findings}
\begin{itemize}
    \item The MOGE distribution retains analytical tractability while offering
          significantly enhanced flexibility compared to the Exponential and GE
          distributions.
    \item The EM algorithm provides an effective estimation technique for the model,
          particularly when closed-form solutions are not available.
    \item Simulation results demonstrate that parameter recovery improves with
          larger sample sizes and appropriate initialization of EM estimates.
\end{itemize}

\section*{Strengths of the MOGE Model}
\begin{itemize}
    \item Ability to model all four major hazard-rate shapes.
    \item Relatively simple and closed-form expressions for PDF and CDF.
    \item Useful for modeling complex or censored lifetime data.
\end{itemize}

\section*{Limitations}
\begin{itemize}
    \item EM algorithm convergence can be slow or sensitive to initial values.
    \item Analytical derivations are more complex than classical models.
    \item Interpretation of the additional parameter $\theta$ may require domain-specific insight.
\end{itemize}

\section*{Future Work}
Potential directions for further research include:
\begin{itemize}
    \item Exploring Bayesian estimation methods for the MOGE distribution.
    \item Extending the model to accommodate covariates or regression structures.
    \item Investigating robust EM initialization strategies to improve convergence.
    \item Comparing MOGE with other flexible lifetime models such as Weibull-Gamma mixtures or log-location-scale families.
\end{itemize}

Overall, the MOGE distribution remains a promising framework for modeling
complex lifetime patterns and provides a strong foundation for further methodological
and applied research in reliability and survival analysis.
