\chapter{Conclusion}

In this study, we examined the Marshall--Olkin Generalized Exponential (MOGE)
distribution in detail, considering both its theoretical construction and the practical
challenges associated with parameter estimation.
By incorporating an additional shape parameter through the Marshall--Olkin method,
the Generalized Exponential distribution gains substantially increased flexibility,
allowing the hazard-rate function to represent a full range of behaviours, including
increasing, decreasing, bathtub-shaped, and inverted-bathtub patterns.

A comprehensive Expectation--Maximization (EM) based estimation framework was
developed using a latent-variable formulation, which significantly simplifies the
maximization step of the likelihood.
The simulation study produced several important findings.
First, the EM algorithm exhibited stable convergence and consistent parameter recovery
when sufficiently large sample sizes were available.
Bias, variance, and mean squared error decreased systematically as sample size increased,
thereby confirming the desirable asymptotic properties of the estimator.
Second, convergence behaviour was smooth and monotonic, in agreement with theoretical
expectations for the EM algorithm.

Nevertheless, the results also revealed notable limitations.
In small-sample settings, parameter estimates became unstable, likelihood surfaces were
irregular, and the EM updates were highly sensitive to initial values.
These findings emphasize the need for caution when interpreting MOGE estimates in the
presence of limited data.
The simulation evidence indicates that the theoretical strengths of the MOGE model can
be fully realized only when sufficiently large datasets are available to support reliable
estimation.

In summary, the MOGE distribution remains a strong and flexible model for lifetime data,
particularly in situations involving non-monotonic hazard structures.
Future research may explore improved initialization strategies, Bayesian estimation
approaches, or model extensions incorporating covariates or dependence structures.
With larger datasets or multi-batch data collection, the MOGE distribution has the
potential to provide meaningful and practical contributions to large-scale reliability
analysis.
