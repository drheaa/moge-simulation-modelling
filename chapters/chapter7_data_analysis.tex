\chapter{Data Analysis}

\section{Introduction}

The real dataset used in this study consists of first goal times recorded in FIFA Club World Cup matches during the 2025 season. The data was sourced from \textit{FootyStats}, a publicly available football statistics platform. Each observation represents the minute at which the first goal was scored by either the home or away team. Matches with no goals were excluded, since the modelling framework requires strictly positive event times.

After preprocessing and deduplication, 29 observations each were obtained for home and away first-goal timings. These goal times form a positive-valued random variable and are therefore suitable for lifetime or survival models such as the Marshall--Olkin Generalized Exponential (MOGE) distribution.

\section{Exploratory Data Analysis}

\subsection{Histogram of First Goal Times}

\begin{figure}[H]
\centering
\includegraphics[width=0.7\textwidth]{histogram_first_goal.png}
\caption{Histogram of first-goal times in FIFA Club World Cup matches.}
\label{fig:hist_goal}
\end{figure}

Figure~\ref{fig:hist_goal} shows that the first goal tends to occur within the first 30 minutes, with a long right tail indicating that late goals occur but are less frequent. This strong skewness suggests that exponential, Weibull, or generalized exponential-type models may be appropriate. The absence of multimodal patterns supports continuous time-to-event modelling.

\subsection{Empirical CDF (ECDF) for Home vs Away First Goals}

\begin{figure}[H]
\centering
\includegraphics[width=0.7\textwidth]{ecdf_home_away.png}
\caption{Empirical Cumulative Distribution Functions (ECDF) for home and away first-goal times.}
\label{fig:ecdf}
\end{figure}

To compare distributional behavior between home-team first goals ($X_1$) and away-team first goals ($X_2$), ECDFs were generated. As seen in Figure~\ref{fig:ecdf}, both curves rise steeply within the first 20--30 minutes, indicating that early goals constitute a large fraction of the observations. The curves flatten after approximately the 40th minute, demonstrating the presence of a long right tail.

The away-team ECDF rises slightly faster than the home-team ECDF, implying marginally earlier scoring on average. However, their convergence at higher quantiles suggests similar overall scoring tendencies.

The nonlinear shapes of the ECDFs contradict the constant hazard rate assumption of the exponential distribution. Instead, the curves imply an increasing hazard rate, motivating the use of more flexible lifetime models such as Weibull or MOGE.

\subsection{Total Time on Test (TTT) Plots}

\begin{figure}[H]
\centering
\includegraphics[width=0.7\textwidth]{ttt_home.png}
\caption{TTT plot for home-team first-goal times.}
\label{fig:ttt_home}
\end{figure}

\begin{figure}[H]
\centering
\includegraphics[width=0.7\textwidth]{ttt_away.png}
\caption{TTT plot for away-team first-goal times.}
\label{fig:ttt_away}
\end{figure}

Figures~\ref{fig:ttt_home} and \ref{fig:ttt_away} show the Total Time on Test (TTT) plots. In both cases, the empirical curves lie below the 45-degree reference line, indicating an Increasing Failure Rate (IFR):

\[
\text{TTT curve below diagonal} \;\Rightarrow\; \text{Increasing Hazard Rate (IFR)}.
\]

This implies that the likelihood of scoring a first goal increases as time progresses. The home-team TTT curve displays slightly stronger convexity, reflecting greater offensive momentum or tactical pressure in home matches—consistent with home-advantage effects.

\section{Model Fitting and Parameter Estimation}

To assess the suitability of the MOGE distribution, three lifetime models were fitted to the home and away datasets:

\begin{itemize}
    \item Gamma distribution (MLE),
    \item Weibull distribution (MLE),
    \item MOGE distribution (EM algorithm due to latent structure).
\end{itemize}

Each model represents different hazard-rate behaviour:

\begin{itemize}
    \item \textbf{Gamma}: flexible but limited for steeply increasing hazards.
    \item \textbf{Weibull}: can model monotonic increasing or decreasing hazards.
    \item \textbf{MOGE}: introduces an additional parameter $\theta$, enabling highly flexible hazard shapes and improved tail modeling.
\end{itemize}

Model performance was compared using:

\[
\begin{aligned}
\text{Log-likelihood}, \\
\text{AIC (penalized model complexity)}, \\
\text{Kolmogorov--Smirnov (KS) statistic}.
\end{aligned}
\]

A better-fitting model exhibits higher log-likelihood, lower AIC, and lower KS statistic.

\section{Goodness-of-Fit Results}

Fitted cumulative distribution functions were plotted against the empirical ECDF. The MOGE distribution showed the closest agreement with observed data, particularly in the early 0--30 minute window where most goals occur. In contrast, both the Gamma and Weibull models underestimated early scoring intensity.

Despite the additional parameter, MOGE achieved superior AIC and KS values, demonstrating that the improvement in fit outweighed the penalty for higher complexity. Its flexible hazard structure accurately represented the increasing probability of scoring near halftime.

\section{Discussion}

The model results indicate that first-goal scoring in football is not memoryless but time-dependent. The increasing hazard rate observed in the data aligns with strategic match dynamics, where offensive intensity typically rises as time progresses.

The MOGE model captures this behavior more effectively than the Weibull or Gamma distributions due to its flexible hazard structure governed by parameter $\theta$. Simpler models with constant or strictly monotonic hazard rates fail to capture the rapid escalation in scoring risk associated with tactical adjustments, psychological momentum, and end-of-half pressure.

Thus, the MOGE model provides a more realistic representation of football scoring processes and is well-suited for predictive analytics.

\section{Conclusion of Real Data Analysis}

This chapter presented a comprehensive analysis of first-goal scoring times using real FIFA Club World Cup match data. The histogram and ECDFs revealed strong right skewness, early scoring concentration, and a long tail of late goals, with more than 80\% of goals occurring before the 40th minute.

TTT plots confirmed an increasing hazard rate for both home and away teams, ruling out constant-rate exponential behavior. Weak correlations between $X_1$ and $X_2$ suggested independence between home and away scoring processes. Bootstrapping further revealed variability in parameter estimation due to sample size constraints.

Together, these findings motivate the simulation study presented in the next chapter and justify the use of advanced lifetime models such as MOGE for understanding and predicting football scoring behavior.