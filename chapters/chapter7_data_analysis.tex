\chapter{Data Analysis}

\section{Objective}

In this chapter, we analyse real tensile-strength data consisting of 56 single carbon
fibre measurements tested under tension at a gauge length of 1~mm, measured in GPa.
The dataset was originally provided by Prof.~R.~G.~Surles and represents strength values
of individual fibres extracted from a 1000-filament tow. Since tensile strength is an
inherently positive, continuous, time-to-failure-type variable, probabilistic modelling
using lifetime distributions is a natural and appropriate choice.

This analysis assesses the suitability of the Marshall--Olkin Generalized Exponential
(MOGE) distribution for modelling material-strength data, in comparison with classical
reliability models such as the Weibull and Gamma distributions. The flexibility of the
MOGE model in capturing increasing hazard structures and tail behaviour is examined
through its three-parameter configuration: $\alpha$, $\lambda$, and $\theta$.

Simulation provides a controlled environment for evaluating estimator performance.
Unlike real experimental data, simulated samples allow true parameter values to be
specified and synthetic observations to be generated directly from the MOGE
distribution. This removes external variability and data limitations, making it
possible to assess how effectively the EM algorithm recovers parameters under ideal
conditions. If the estimator converges toward the true parameter values as sample size
increases, then the MOGE model can be regarded as theoretically sound, even when real
applications remain challenging due to limited data availability.

% ---------------------------------------------------------
\section{Simulation Setup and Methodology}

Synthetic i.i.d.\ samples were generated from the MOGE distribution with true parameter
values
\[
(\alpha_0 = 1.5,\; \lambda_0 = 0.8,\; \theta_0 = 1.2),
\]
chosen to reflect moderately increasing hazard rates commonly observed in material
failure processes.

Samples were generated for increasing sample sizes
\[
n \in \{30,\, 50,\, 100,\, 200\}.
\]

For each sample size, the estimation procedure was repeated across multiple Monte Carlo
replications. In each run, synthetic failure times were drawn from the MOGE model, the
EM algorithm was applied to re-estimate parameters, and the resulting estimates were
recorded. This structure allowed for a systematic evaluation of estimator accuracy,
variability, and convergence behaviour as data availability increased.

% ---------------------------------------------------------
\section{Boxplots of Parameter Estimates}

\begin{figure}[htbp]
\centering
\includegraphics[width=0.85\linewidth]{figures/fig7_1_boxplot.png}
\caption{Boxplots of EM parameter estimates across different sample sizes.}
\label{fig:boxplots}
\end{figure}

The distribution of parameter estimates for each sample size is illustrated in
Figure~\ref{fig:boxplots}. When $n=30$, estimates are widely dispersed, indicating high
variance and substantial uncertainty. As sample size increases to $n=50$ and $n=100$,
the spread of estimates narrows considerably, with median values moving closer to the
true parameter values. By $n=200$, the estimator becomes tightly concentrated, indicating
strong convergence and stable parameter recovery.

These results demonstrate that the MOGE estimator is sensitive to small samples but
stabilises rapidly once sufficient data are available.

% ---------------------------------------------------------
\section{Mean Squared Error (MSE) vs Sample Size}

\begin{figure}[htbp]
\centering
\includegraphics[width=0.85\linewidth]{figures/fig7_2_mse_plot.png}
\caption{Mean squared error of EM parameter estimates as a function of sample size.}
\label{fig:mse}
\end{figure}

The mean bias and mean squared error were calculated for each estimated parameter.
As shown in Figure~\ref{fig:mse}, both metrics exhibit a clear decreasing trend as sample
size increases. For small samples, parameter estimates display higher variability and
systematic deviation from the true values. As data volume grows, both bias and MSE
decline sharply, confirming that the EM estimator for MOGE is asymptotically consistent
and increasingly precise with larger samples.

% ---------------------------------------------------------
\section{Convergence Behaviour of the EM Algorithm}

\begin{figure}[htbp]
\centering
\includegraphics[width=0.85\linewidth]{figures/fig7_3_convergence.png}
\caption{Convergence behaviour of the EM algorithm for a simulated sample ($n = 100$).}
\label{fig:convergence}
\end{figure}

To assess numerical stability, log-likelihood values were recorded across EM iterations.
Figure~\ref{fig:convergence} illustrates the improvement in likelihood following the EM
update step. As expected from EM theory, the likelihood increases monotonically after
each iteration, reflecting the algorithm’s non-decreasing likelihood property.

Although the displayed trace focuses on the initial update for clarity, full
multi-iteration runs showed continued likelihood improvement until convergence,
confirming smooth and stable behaviour without oscillation or divergence when adequate
data are available.

% ---------------------------------------------------------
\section{Bootstrapping for Parameter Stability}

To further assess estimator reliability under limited data conditions, a non-parametric
bootstrap procedure was applied. Bootstrap samples were generated with replacement from
the observed tensile-strength data, and the MOGE model was re-fitted using the EM
algorithm for each resample.

\begin{figure}[htbp]
\centering
\includegraphics[width=0.85\linewidth]{figures/fig7_4_boot_hist.png}
\caption{Bootstrap histograms for $\alpha$, $\lambda$, and $\theta$.}
\label{fig:boot-hist}
\end{figure}

\begin{figure}[htbp]
\centering
\includegraphics[width=0.75\linewidth]{figures/fig7_5_boot_box.png}
\caption{Bootstrap-based variability of MOGE parameter estimates.}
\label{fig:boot-box}
\end{figure}

Figure~\ref{fig:boot-hist} reveals wide and skewed bootstrap distributions, highlighting
substantial estimation uncertainty when modelling small datasets. This variability is
further illustrated in Figure~\ref{fig:boot-box}, where broad interquartile ranges and
shifting medians indicate limited parameter stability. These results reinforce the
simulation findings that small sample sizes restrict reliable inference for the MOGE
model.

% ---------------------------------------------------------
\section{Scatter Plot Analysis for Dependency Assessment}

\begin{figure}[htbp]
\centering
\includegraphics[width=0.7\linewidth]{figures/fig7_6_scatter.png}
\caption{Scatter plot of paired tensile-strength observations.}
\label{fig:scatter}
\end{figure}

Figure~\ref{fig:scatter} shows no visible clustering or monotonic trend between paired
observations. This lack of structure is supported by correlation coefficients
\[
\text{Pearson}=-0.0376,\qquad
\text{Spearman}=-0.0992,\qquad
\text{Kendall}=-0.0695,
\]
all of which are close to zero. These results indicate negligible dependence, supporting
the assumption that observations may be modelled independently within the lifetime
framework.

% ---------------------------------------------------------
\section{Conclusion}

The analysis conducted in this chapter provides strong evidence for the theoretical
robustness of the Marshall--Olkin Generalized Exponential distribution under controlled
data conditions. Simulation results demonstrate that EM-based parameter estimation
exhibits stable convergence and consistent recovery of true parameter values as sample
size increases, with both bias and mean squared error decreasing systematically.

Bootstrap diagnostics further highlight the limitations imposed by small sample sizes,
where parameter uncertainty remains substantial. Overall, these findings confirm that
the MOGE model is statistically sound and performs reliably when sufficient data are
available, while emphasising the need for caution when applying the model to sparse
experimental datasets.
