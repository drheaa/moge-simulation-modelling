\begin{abstract}
This project presents a focused study of the Marshall--Olkin Generalized Exponential (MOGE)
distribution, a flexible three-parameter lifetime model that extends the Generalized
Exponential family through the Marshall--Olkin method.
The model’s ability to represent a wide variety of hazard-rate shapes---such as increasing,
decreasing, bathtub, and inverted-bathtub patterns---motivates its use in reliability
contexts where classical models prove insufficient.

A complete likelihood-based estimation procedure is developed using the
Expectation--Maximization (EM) algorithm, with particular attention to the latent-variable
structure that enables tractable parameter updates.
Incorporating fixed-point iterations for two of the parameters and a closed-form update
for the third produces a practical and stable estimation routine.
A comprehensive simulation study evaluates estimator behaviour across multiple sample
sizes, demonstrating that parameter recovery improves markedly as data volume increases.
The anticipated EM pattern of non-decreasing likelihood is established through
convergence diagnostics, and the model’s theoretical consistency is confirmed using bias
and mean squared error analyses.

Nevertheless, the results indicate that sensitivity to small sample sizes is inevitable,
as likelihood surfaces become unstable and parameter estimates vary widely.
Collectively, these findings suggest that the MOGE distribution is a powerful and
flexible lifetime model; however, its effective application relies on sufficient data
availability and thoughtful initialization of the EM algorithm.
\end{abstract}
